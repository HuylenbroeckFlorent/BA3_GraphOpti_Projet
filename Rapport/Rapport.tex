\documentclass[b]{beamer}

\usepackage[french]{babel}
\usepackage[utf8]{inputenc}

\usepackage{graphicx}
\usepackage{amsmath,amsfonts,amsthm}

\usetheme{Warsaw}

\title[Projet de Graphes et Optimisation Combinatoire]{Projet de Graphes et Optimisation Combinatoires}
\author{DACHY Corentin, HUYLENBROECK Florent, JOSSE Thomas}
\begin{document}

\begin{frame}
	\titlepage
\end{frame}

\begin{frame}
	\frametitle{Table des Matières}
	\tableofcontents
\end{frame}

\section{Méthodes Débattues}

\begin{frame}
	
	3 méthodes débattues:
	\begin{itemize}
		\item GRASP
		\item Recherche Taboue
		\item Colonie de Fourmis
	\end{itemize}
	2 Choisies :
		\begin{itemize}
			\item Recherche Taboue
			\item Colonie de Fourmis
		\end{itemize}
	1 Implémentée:
		\begin{itemize}
			\item Colonie de Fourmis Hybride
		\end{itemize}
\end{frame}

\subsection{GRASP}

\begin{frame}

\begin{itemize}
	\item Idée Générale
	\begin{enumerate}
		\item Glouton avec Proba
		\item Recherche locale 
	\end{enumerate}
	\item Pourquoi? 
	\begin{enumerate}
		\item Point de départ pour autres algos
		\item Abandonné après début colonie de fourmis hybride
	\end{enumerate}
\end{itemize}

\end{frame}

\subsection{Recherche Taboue}

\begin{frame}

	\begin{itemize}
		\item Idée Générale\\
		\begin{enumerate}
			\item Recherche locale avec liste d'objets parcourus 
		\end{enumerate}
		\item Pourquoi?\\
		\begin{enumerate}
			\item Trouver une permutation de départ efficace
		\end{enumerate}
	\end{itemize}

\end{frame}

\subsection{Colonie de Fourmis}

\begin{frame}

	\begin{itemize}
		\item Idée Générale\\
		\begin{enumerate}
			\item Envoyer des fourmis
			\item Phéromones\\
			\item Diversification et Intensification\\ 
		\end{enumerate}
%			
		\item Pourquoi?\\
	\end{itemize}
\end{frame}

\section{Méthode impléméntée}
\begin{frame}
	L'implémentation finale consiste en une hybridation de l'algorithme de la colonie de fourmis avec l'algorithme de la recherche taboue et, lorsque la taille est trop grande, d'une recherche locale.
\end{frame}
\end{document}
