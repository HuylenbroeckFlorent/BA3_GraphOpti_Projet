\documentclass[b]{beamer}

\usepackage[french]{babel}
\usepackage[utf8]{inputenc}

\usepackage{graphicx}
\usepackage{amsmath,amsfonts,amsthm}

\usetheme{Warsaw}

\title[Projet de Graphes et Optimisation Combinatoire]{Projet de Graphes et Optimisation Combinatoires}
\author{DACHY Corentin, HUYLENBROECK Florent, JOSSE Thomas}
\begin{document}

\begin{frame}
	\titlepage
\end{frame}

\begin{frame}
	\frametitle{Table des Matières}
	\tableofcontents
\end{frame}

\section{Méthodes Débattues}

\begin{frame}
	
	3 méthodes débattues:
	\begin{itemize}
		\item GRASP
		\item Recherche Tabou
		\item Colonie de fourmis
	\end{itemize}
	2 Choisies:
		\begin{itemize}
			\item Recherche Tabou
			\item Colonie de fourmis
		\end{itemize}
	1 Implémentée:
		\begin{itemize}
			\item Colonie de fourmis hybride
		\end{itemize}
\end{frame}

\subsection{Recherche Tabou}
\begin{frame}
	\begin{itemize}
		\item Idée Générale\\
		\begin{enumerate}
			\item Voisinage = toute permutation de 2 éléments.\\
			(Ex: Pour [1,2,3] le voisinage serait : [[2,1,3],[3,2,1],[1,3,2]])
			\item Deux types de recherche tabou: par mouvement et par solution
			\item Longueur de la liste Tabou : n (+x) mais décroisant
			\item La recherche effectue $N$ itérations en mouvement, ensuite, on effectue une recherche par solution jusqu'à ce que l'on n'améliore plus notre solution.
		\end{enumerate}
		\item Gardée pour l'implémentation finale du projet (Voir slide 7)
	\end{itemize}
\end{frame}


\section{Méthode impléméntée}
\subsection{Fourmis pour les QAPs}
\begin{frame}

Afin d'appliquer la colonie de fourmis aux problèmes d'affectation quadratiques, il faut voir la matrice des phéromones non plus comme "La probabilité de passer de $i$ à $j$" mais plutôt "La désirabilité que $s_i=j$ dans notre solution $s$ ". \\
Les fourmis ne construisent plus une solution à partir d'un point de départ, mais explorent le voisinage d'une solution de départ.\\
\end{frame}

\subsection{La colonie de fourmi hybride}
\begin{frame}
L'hybridation apportée à cette colonie de fourmi consiste en deux points :\\
	\begin{itemize}
	\item Les solutions initiales sont affinées par une recherche locale (\emph{fast-start}) ou par une recherche tabou (\emph{slow-start})\\
	\item Chaque fourmi, lors d'une itération, applique elle-même une recherche locale à sa solution modifiée.\\
	\end{itemize}
Et fonctionne selon deux modes :
	\begin{itemize}
	\item Une phase d'intensification : les fourmis gardent toujours la meilleure de leurs solutions.
	\item Une phase hors intensification : les fourmis gardent toujours la solution générée par leurs permutations.\\[.5cm]
	\end{itemize}
Une phase plus rare de diversification vient parfois s'ajouter.
\end{frame}

\subsection{Fonctionnement}
\begin{frame}
Lors d'une itération, une fourmi $k$ applique à sa solution $s^k$ :
\begin{itemize}
	\item Un certain nombre de permutations basées sur la matrice des phéromones $\rightarrow s^{'k}$.
	\item Une recherche locale $\rightarrow s^{''k}$.
	\item Si intensification : $s^{''k}\leftarrow$max($s^k$,$s^{''k}$)
\end{itemize}
Ensuite, après évaporation des phéromones (par un facteur $\alpha=0.9$) seule la meilleure fourmi dépose les siennes, ce qui rend la recherche assez intense.\\
Pour contrer cela, si un certain nombre d'itérations se sont déroulées sans améliorer la meilleure solution $s^*$, une remise à zéro des solutions tenues par les fourmis est effectuée, en ne gardant que la meilleure solution $s^*$ dans une fourmi. Une recherche tabou est ensuite effectuée à la place d'une recherche locale pour la première nouvelle itération.\\[.5cm]
Le nombre de fourmis est fixé à $10$.
\end{frame}

\subsection{Implémentation}
\begin{frame}
En ce qui concerne l'implémentation, les fourmis nous semblaient avantageuses pour une raison, la possibilité de lancer chaque fourmi dans un thread. Les étapes lors de l'implémentation ont été :
\begin{itemize}
	\item Premier jet en python3 (Colonie de fourmi + recherche locale).
	\item Traduction en $C$++.
	\item Ajout du \emph{multi-threading}.
	\item Ajout de la recherche tabou lors de l'initialisation.
\end{itemize}
Chaque étape ayant mené à une amélioration de la vitesse ou de la qualité de la solution trouvée.
\end{frame}
\end{document}
